In this project, we gave readers a thorough overview of Secondary Frequency Control and PI control: the algorithms (PART I) and the test case scenario (Nordic) (PART II), as well as how we contributed to the analytical tools to connect the theories and testings. 

In Chapter~\ref{Chapter2}, we walked through the overview of Frequency Control, including Primary Frequency Control, Secondary Frequency Control and Tertiary Frequency Control. We knew the reasons we need Frequency Control and the mechanisms of the three Frequency Controls. Equally important is we knew the response time and the feature of the three Frequency Controls. 

In Chapter~\ref{Chapter3}, we covered the theory elements of Secondary Frequency Control and PI control. We introduce the physical theory behind Secondary Frequency Control and the logic of a central control algorithm. We explained we need to simplify the model to remove some conditions that not effect the simulation results. We introduce the mathematical and physical theory of PID control. Through the frequency problems in the reality, we understood the reasons using P-term and the I-term in SFC and the reasons not using D-term because of amplifying the high component parts in the signal. Then, we introduced the core codes in PI control in Python and how we send the information into the generators to fix the frequency problems. Besides, we introduced the deadband error and discussed the deadband control algorithm. In the last part of this chapter, we introduced our tuning methodology. First, we analysed the impacts of ki and then we introduced bisection method into tuning models. Then, we introduced analytical models written in Python and MATLAB and finally they can be used in filtering unacceptable results according to signal requirements, in generating 2D diagram, and in generating 3D triangle surface plot with multiple needs. 

In PART II, the key questions we wanted to answer are: Is there any feature of a PI control under the condition of low time delay? What are the impacts of time delay to such a PI controller? Where is the best tuning result? How can we use the idea of Emergency Control to avoid emergency situations? 

In Chapter~\ref{Chapter4}, we put forward how to assume the conditions, i.e. hypothesis, of the Nordic system and finally, we choose suitable value of kp, ki, delay, start time, end time etc. Then, we discussed the expected outcome based on physical theory of SFC before the simulation starts. Then, we introduced how to tune PI control and finally we showed you the simulation results. We think a larger kp take a faster response but at the same time it has a chance to exceed the limit of speed of power output. 

In Chapter~\ref{Chapter5}, like in Chapter 4, we put forward how to assume the conditions of the Nordic system and finally, we choose the range of time delay and keep the condition in Chapter 4. Then, we discussed the expected outcome based on physical theory of SFC before the simulation starts. Then, we introduced how to tune time delay and finally we showed you the simulation results. We think there are some outliers that will interference the analysis so we decided to filter them. We concluded that acceptable results is less and kp and ki are shrank to each other with a larger time delay. 

In chapter~\ref{Chapter6}, we discussed the impact of time delay under the condition of  Emergency Control. We concluded that there is no different for Emergency Control and a standard PI control interns on the impact of time delay. Besides, we introduced how to use the idea of Emergency Control to avoid emergency situations by limiting the speed of power output. We gave the mathematics equation and the related Python codes. Finally, we decide to reference the rule in the United States to limit the speed. 

I am really excited about the progress that has been made for the past one academic year. At the same time, we also deeply believe there is still a long way to go towards smart grid algorithms, and we are still facing challenges and a lot of open questions that need to address in the future. One key challenge is that we still do not have a better way to analyse the speed of power out if the ramp time is so small because, basically, the computer program give a feedback of zero. Normally, the problems like this be unnoticed in big data simulations. 