\section{Overview} %6.1

Kundur and Morison discussed about emergency control in \sys{Techniques for emergency control of power systems and their implementation}. In that paper, they introduced alert state before describing emergency control. They think, \cite{kundur1997techniques} an alert state will be entered when if the security level is below "a certain limit of adequacy", or if the probability of heavy disturb increases due to the weather conditions. However, in an alert state, all constraints are met. Emergency state is at state that one or more constraints are not met. Thus, the emergency control tries to do prevent the emergency state happening and an emergency control will triggered at the alert state.

In all, emergency control should be established in emergency conditions to minimise the risk of further uncontrolled separation, loss of generation, or system shutdown. In reality, \cite{kundur1997techniques} alert thresholds can be established. If the system exceeds one of the alert thresholds, emergency control can start.

In our case, emergency control starts from the lowest point of frequency in Primary Frequency Control and use Secondary Frequency Control to restore its frequency to prevent the risk of black out.