\section{Thesis Outline} %1.2
This project consists of two parts: PART I Algorithms and PART II Test Case Scenario (Nordic).\\

PART I focuses on the task of understanding and building Secondary Frequency Control (SFC) model and PI control algorithm so that we are able to test our cases in PART II.\\


In Chapter~\ref{Chapter2}, we give an overview of Frequency Control, including Primary Frequency Control (PFC), Secondary Frequency Control (SFC) and Tertiary Frequency Control (TFC).\\

In Chapter~\ref{Chapter3}, we formally focus on the physical theory behind Secondary Frequency Control (SFC) which is the base of the whole project. We briefly discuss PID control and argue that we should use PI control to reduce the probability of risk. We then discuss how to build a communication layer on top of an existing smart grid simulator and how to design a centralised controller for stabilising the system. We describe the algorithms we built, the tuning methodology, and some implementations. We finally define the acceptable results based on the official report from Nordic, and with that, we build our own analytical tools. With these tools, we can plot a 2D and even a 3D diagram, find the eligible results and give feedback to our controller in PART II.\\

PART II views testing in a specific test case scenario (Nordic) as an important part such as the impact of different time delays. Detailedly,\\

In Chapter~\ref{Chapter4}, we focus on testing the system in a low time delay. We discuss how to choose an appropriate generator as a breaker and how to tune the range of  gain. Before simulating, we predict some expected results based on the physical theory behind Secondary Frequency Control (SFC). Then we present a comprehensive evaluation on the simulation results. We describe the results and compare them with the expected one. We discuss why my prediction had deviation or missing. We finally show a 2d plot and a simulation result.\\

In Chapter~\ref{Chapter5}, we discuss the impact of different time delays. We increase the delay and use the range of kp and ki in Chapter~\ref{Chapter4}. We explain why we think it is reasonable to continue using the range of gain in Chapter~\ref{Chapter4}. Then we predict the simulation results based on both the physical theory behind Secondary Frequency Control (SFC) and the results in Chapter~\ref{Chapter4}. Then we present a comprehensive evaluation on the complicated simulation results. We describe the results with a plotted 3d graph. We discuss some unpredictable results and some seemingly irregular data. We discuss the risk of the generators, like what we do in Chapter~\ref{Chapter4}, and how to remove unacceptable results. We  finally show a 3d plot and a best simulation result.\\

In Chapter~\ref{Chapter6}, we test Emergency Control and discuss the impact of different time delays without tuning gains. We firstly define Emergency Control although we have discussed it in Chapter~\ref{Chapter2}. We predict some results based on Secondary Frequency Control (SFC) and the conclusions in Chapter~\ref{Chapter4}. After implementing and analysing the results, we discuss why my predictions are “wrong”. We finally do a risk assessment for the system as we did before.\\


We finally conclude in Chapter~\ref{Chapter7}.\\